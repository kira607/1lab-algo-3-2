\section{Описание реализуемого класса и методов}

Задание было выполнено на языке \verb|Python| версии 3.8.

Реализация выполнена в виде трёх классов: 
\verb|RBTreeColor|, \verb|RBTreeNode|, и \verb|RBTree|.

\subsection{RBTreeColor}

\verb|RBTreeColor| реализует цвет узла дерева.

Цвет может быть либо красным, либо чёрным.

\subsection{RBTreeNode}

\verb|RBTreeNode| реализует узел дерева.

Узел дерева имеет ключ, значение, цвет и ссылки на
левого, правого потомка и родительский узел.
Сравнение узлов происходит через "\verb|python magic methods|"
(
\verb|__eq__|, 
\verb|__lt__|, 
\verb|__gt__|, 
etc.
),
где сравниваются хэши ключей.

Так же у узла можно получить:

\begin{itemize}
    \item "дедушку" (предка воторого поколения) через \verb|property| метод \verb|gparent|.
    \item "дядю" (узел, соседний с родительским) через \verb|property| метод \verb|uncle|.
    \item "брата" (соседний узел) через \verb|property| метод \verb|bro|.
\end{itemize}

Узел можно перекрасить методами 
\verb|color_black()| и \verb|color_red()|.

А так же получить информацию о расположении и цвете:
\verb|is_black()|,
\verb|is_red()|,
\verb|is_left()|,
\verb|is_right()|.

\subsection{RBTree}

\verb|RBTree| реализует само красно-чёрное дерево.

Реализация класса выполнена в 
"\verb|pythonic way|" с помощью 
"\verb|python magic methods|".
При этом методы требуемые по заданию были сохранены.

Далее идёт список с кратким описанием каждого метода

\begin{itemize}
    \item \verb|__init__| --- Инициализатор.
    \item \verb|__getitem__| --- Получение значения по ключу.
    \item \verb|__setitem__| --- Создание новой пары либо перезапись существующей.
    \item \verb|__delitem__| --- Удаление пары по ключу.
    \item \verb|__len__| --- Получение количества элементов.
    \item \verb|__str__| --- Строковая реперзентация объекта.
    \item \verb|__iter__| --- Получение итератора (по ключам).
    \item \verb|__bool__| --- Булева репрезентация объекта (False, когда пуст, иначе True)
    
    \item \verb|height| --- Высота дерева
    
    \item \verb|get| --- Безопасное получение значения по ключу, с возможностью указать значение по умолчанию.
    \item \verb|items| --- Генератор по ключам и значениям одновременно
    \item \verb|keys| --- Получить список ключей.
    \item \verb|values| --- Получить список значений.
    \item \verb|print_tree| --- Вывести объект в виде дерева.
    \item \verb|get_dot_string| --- Получить описание дерева на языке DOT (для визульной репрезентации).
    
    \item \verb|insert| --- Создание новой пары либо перезапись существующей.
    \item \verb|remove| --- Удаление пары по ключу.
    \item \verb|find| --- Получение значения по ключу.
    \item \verb|clear| --- Очистка дерева.
    \item \verb|get_keys| --- Получить список ключей.
    \item \verb|get_values| --- Получить список значений.
    \item \verb|print| --- Вывести строуовую репрезентацию объекта.
    
    \item \verb|_get_max_node| --- Получить узел с максимальным значением ключа.
    \item \verb|_get_min_node| --- Получить узел с минимальным значением ключа.
    \item \verb|_get_height| --- Получить высоту поддерева для узла.
    
    \item \verb|_fix_insert| --- Исправление вставки.
    \item \verb|_insert_case_1| --- Исправление вставки, случай 1.
    \item \verb|_insert_case_2| --- Исправление вставки, случай 2.
    \item \verb|_insert_case_3| --- Исправление вставки, случай 3.
    \item \verb|_insert_case_4| --- Исправление вставки, случай 4.
    \item \verb|_insert_case_5| --- Исправление вставки, случай 5.
    
    \item \verb|_replace| --- Заменить узел потомком.
    \item \verb|_delete| --- Удалить узел (с $ \le 1 $ потомком)
    
    \item \verb|_fix_delete| --- Исправление удаления.
    \item \verb|_del_case_1| --- Исправление удаления, случай 1.
    \item \verb|_del_case_2| --- Исправление удаления, случай 2.
    \item \verb|_del_case_3| --- Исправление удаления, случай 3.
    \item \verb|_del_case_4| --- Исправление удаления, случай 4.
    \item \verb|_del_case_5| --- Исправление удаления, случай 5.
    \item \verb|_del_case_6| --- Исправление удаления, случай 6.
    
    \item \verb|_traverse_preorder| --- Префиксный обход узлов
    \item \verb|_traverse_inorder| --- Инфиксный обход узлов.
    \item \verb|_traverse_postorder| --- Постфиксный обход узлов.
    
    \item \verb|_left_rotate| --- Левый поворот.
    \item \verb|_right_rotate| --- Правый поворот.
    
    \item \verb|_print_tree| --- Рекурсивный метод вывода дерева.
    \item \verb|_get_node| --- Получить узел по ключу.
    \item \verb|_get_leaf| --- Получить узел-родитель по вствляемому ключу.
    \item \verb|_swap_kv| --- Поменять местами данные в узлах.
\end{itemize}

